"约翰·塞巴斯第安·巴赫在1748年。取自艾利亚斯·哥特利伯·豪斯曼的一幅油画。",
"无忧宫中的长笛音乐会,阿道夫·封·门采尔\pnum{1852}。",
"\label {fig:3}国王主题。",
"(奉旨承诏,将歌曲及余部以卡农技巧予以解决。)巴赫在这里使用卡农(canonic)这个词语意双关,它不仅有“用卡农(with canon)”的意思,还包含着“用可能有的最好方式”的意思。这句题辞的每个词的首字母排在一起是",
"\label {fig:5}瀑布,艾舍尔作(石版画,1961)。",
"\label {fig:6}上升与下降,艾舍尔作(石版画)。",
"举着反光球的手。毛里茨·康奈利斯·艾舍尔的自画像(蚀版画,l935)。",
"变形II,艾舍尔作(木刻,19.5×400厘米,1939--1940)。",
"库特·哥德尔。",
"莫比乌斯带I,艾舍尔作(四版套印木刻,1961)。",
"一个系统地构造出来的WJU系统中所有定理的“树”。向下到第N步,它包含了恰恰需要N个步骤推导的那些定理.圆圈中的数字字表示使用了哪条规则。WU是不是在这棵树上的什么地方?",
"空中城堡,艾舍尔作(木刻,1928)。",
"释放,艾舍尔作(石版画,1955)。",
"镶嵌画II,艾舍尔作(蚀版画,1957)。",
"刘皓明绘。",
"以鸟作瓦,艾舍尔作(选自1942年的速写本)。",
"FIGURE - FIGURE图形,斯科特·凯姆作\pnum{1975}。你能看出其中的单词吗?",
"\label {fig:18}这张图形象地描绘了TNT中各种符号串之间的关系。最大的方框代表所有TNT符号串的集。次一级的方框代表所有良构的TNT符号串的集,在它里面有TNT的所有语句的集。现在事情开始有趣了。所有定理的集合画成了一棵树,树干代表公理的集合。选树来作象征是因为它展现出来的递归生长的模式:新的枝条不断地从旧的里面生长出来。手指一样的枝条探到了约束区域(真理集)的各个角落,然而永远无法完全覆盖这个区域。真理和“假理”之间的分界线有意地提示出一种曲折随机的海岸线,无论你如何仔细地察看,总是有更细微的结构,因而不可能用有穷的方式精确地加以描述。(参见孟德尔布劳特所著《碎片》。)对应的树代表定理的否定的集:它们都是假的,但它们整个却无法充满假陈述的空间.[作者绘]",
"巴赫《赋格的艺术》最后一页。在原手稿上,巴赫的儿子卡尔·菲利普·埃玛努厄尔写道:“请注意,在这支赋格中,当名字B. A. C. H.作为对题被引入时,作者死去了”(方框标出的即是B-A-C-H。)我把巴赫的这最后一支赋格的最后一页当作他的墓志铭。[乐谱由唐纳德·伯德的程序“斯马特”印出.该程序是在印第安纳大学开发的。]",
"哥德尔定理背后的原理的形象化表现:两个映射背靠背具有了出乎意料的飞去来器效果。第一个映射是从纹道模式到声音,由一个唱机实现。第二个——很平常,但却常常被忽视——是从声音到唱机的震颤。注意第二个映射独立于第一个映射而存在,因为附近的任何声音——不仅是唱机自己产生的声音——都会引起这种震颤。哥德尔定理转述在这里就是:对任何一个唱机,都有它不能播放的唱片,因为后者会导致前者的间接自摧毁。[作者绘。]",
"巴别塔,艾舍尔作(木刻,1928)。",
"相对性。艾舍尔作(版画,1953)。",
"\label {fig:23}凹与凸,艾舍尔作(石板画,1955)。",
"爬虫,艾舍尔作(石板画,1943)。",
"克里特迷宫(意大利蚀刻画:费尼盖拉派)[选自马修斯[W. H. Mathews]所著《迷津和迷宫——其历史及发展》[Mazes and Labyrinths: Their History and Development]纽约:Dover Publications,1970年版。]",
"对话《和声小迷宫》结构示意图。垂直下落是“推入”,上升是“弹出”。注意此图与对话中缩进格式的相似之处。从这个图可以清楚地看出最初的紧张——赫晕的恫吓——从未解决,阿基里斯和乌龟还在空中悬着。有些读者可能会为这个没弹出来的推入而备受折磨,而有些读者则可能眼都不眨。在这个故事里,巴赫的音乐迷宫也同样被过于迅速地截止——而阿基里斯甚至没有注意到任何怪异之处。只有乌龟意识到了那个全局性的由于悬搁而造成的紧张。",
"花哨名词与豪华名词的递归迁移网。",
"豪华名词的RTN,其中有一个结点递归地扩展了。",
"(a)图案G,未扩展。\hspace {3em}(b)图案H,未扩展。\protect \ (c)图案G,扩展一次。\hspace {2.25em}(d)图案H,扩展一次。",
"图案G,进一步地扩展了,并为结点标了数字。",
"一个斐波那契数的递归迁移网。",
"\label {fig:32}函数INT(x)的图形。在x的每个有理数值处都是个跳跃性的不连续点。",
"\label {fig:33}(a)能用递归替换构造INT的骨架。",
"\label {fig:34}G图:一个递归图形,显示了一个磁场中理想晶体里的电子的能量带。$\alpha $代表磁场强度,沿垂直方向从0到1。能量是沿水平方向的。水平的线段是所允许的电子能量带。",
"\textcircled {\tiny {5}}一个费因曼图案,表示出重正化了的电子从A到B的传播。图中,时间从左向右增加,因此,在电子的箭头指向左边的片段中,它的移动是“与时间反向的”。更为直观的说法是,反电子(正电子)在正向的(即正常的)时间中移动。光子的反粒子是它们自己,因此它们的曲线无须箭头。",
"色和鳞,艾舍尔作(木刻,1959)。",
"蝴蝶,艾舍尔作(木刻,1950)。",
"\label {fig:38}三子棋开始两步的超前搜索树。",
"罗塞达碑",
"文字集。",
"\label {fig:41}这个巨大的非周期性晶体结构是噬菌体$\varphi X174$染色体的碱基序列。它是所有生物中第一个被完整地给出的染色体组。如果要展示争个大肠杆菌细胞的碱基序列,就需要用人约2000页纸来写这种犁耕体。展示单个人体细胞DNA的碱基序列大约需要一百万页纸。现在你手中这本节所包含的信息量差小多等于描写一个渺小的大肠杆菌细胞的结构的信息量。",
"\label {fig:42}“螃蟹卡农”,艾舍尔作。",
"\label {fig:43}这里是螃蟹的一种基因中的一小段,它们一圈圈地缠绕着。当两股DNA链被拆开并且并排铺开时,它们读起来是这样的:\protect \ …TTTTTTTTTCGAAAAAAAAA…\protect \ …AAAAAAAAAGCTTTTTTTTT…\protect \ 请注意它们是一模一样的,只是两者的走向正相反,一个前行时,另一个正好倒行。这正是音乐中所谓的“螃蟹卡农”这种形式所具有的特征。虽然不尽相同,它还是令人联想起回文来。回文就是一种正着读和倒着读都一模一样的句子。在分子生物学中,这样的DNA片段被称为“回文”——这种称呼有点用词不当,因为称它为“螃蟹卡农”要来得更准确些。不仅这个DNA片段是这样——而且本篇对话结构编排的基本序列也是螃蟹卡农式的。注意,在英文中,“阿基里斯[Achilles]”的第一个字母是A,“乌龟[Tortoise]”是T,“螃蟹[Crab]”是C,G则是“基因[Gene]”的第一个字母。",
"《螃蟹卡农》,选自巴赫《音乐的奉献》。[乐谱是用唐纳德·伯德的程序“斯马特”印制的,美术字由陈春光设计。]",
"清真寺,艾舍尔作(黑白粉笔画,1936)。",
"\label {fig:46}三界,艾舍尔作(版画,1955)",
"\label {fig:47}露珠,艾舍尔作(镂刻凹版,1947)。",
"\label {fig:48}另—个世界,艾舍尔作(木雕版,1947)。",
"\label {fig:49}白天与黑夜,艾舍尔作(木刻,1938)。",
"\label {fig:50}果皮,艾舍尔作(木雕,1955)。",
"\label {fig:51}泥塘,艾舍尔作(木刻,1952)。",
"\label {fig:52}水面涟漪,艾舍尔作(油毡浮雕,1950)。",
"\label {fig:53}三个球II,艾舍尔作(版画,1946)。",
"莫比乌斯带II,艾舍尔作(木刻,1963)。",
"彼埃尔·德·费马",
"魔带和立方架,艾舍尔(蚀版画,1957)。",
"“组块化”的想法:一组对象被当作一个“块”而重新看成是一个单位。其边界有点象细胞膜或国境线:它使得其内部的那一组对象构成一个独立的单位。根据不同的需要,可以考虑组块的内部结构,也可以不考虑。",
"汇编程序和编译程序都是用于向机器语言进行翻译的程序。这用直线表示。它们自身也是程序,开始时也需用某种语言写成。波浪线表示编译程序可以用汇编语言书写,汇编程序可以用机器语言书写。",
"为了构造智能程序,需要建造一个硬件和软件的层次结构,以解除在最低层观察一切所带来的痛苦。对同一个过程,不同层次的描述会是很不一样的,只有顶层是充分地组块化了,因此容易被人所理解。[摘自帕·亨·温斯顿,《人工智能》]",
"\label {fig:60}刘友庄绘。",
"“蚂蚁赋格”,艾舍尔作(木刻,1953)。",
"刘皓明绘。",
"徙途中,兵蚁有时会用它们自己的身体搭起一座活桥。在这幅照片中就是这样一座桥。可以看到一个蚁群的工蚁们腿脚相连,它们的跗节肢钩在一起,顺着桥头形成许多不规则的链系统,一只共生的蠢鱼正在过桥,此时它处在中心部位。引自爱德华·威尔逊的《昆虫社会》!",
"刘皓明绘。",
"神经原示意图[摘自伍尔德里奇[D.Wooldridge],《大脑的机制》[TheMachinery of the Brain],纽约:Mc-Graw-Hill,1963年版,第6页。]",
"人脑左视图。奇怪的是视觉区在头的后部。[摘自史蒂文·罗斯的《有意识的大脑》,修订本,第50页。]",
"某些神经原样本对模式的反应。\protect \ (a)这个负责边缘检测的神经原寻找左明右暗的垂直边缘。第一列显示了边缘的倾角与该神经原的关系,第二列说明边缘在域中的位置与这个特定神经原无关。\protect \ (b)显示了超复杂细胞的更具选择性的反应:在这里仅当下行的舌状物通过域中间时反应最强。\protect \ (c)假想的“祖母细胞”对各种随机刺激的反应。读者不妨思考一下对同一组刺激一个“章鱼细胞”将作何反应。",
"在这张示意图中,假设神经原成点状分布在一个平面上。两条互相交叉的神经通道用不同的灰度来表示。可能出现这样的情况:两个独立的“神经闪电”同时通过这两条通道,像池塘水面上的两个涟漪一样彼此穿过(见图\ref {fig:52})。这形象地说明了两个共享神经原的“活跃符号”是如何可能同时被激活的。[摘自约翰·西·艾克勒斯[John C.Eccles],《面对现实》[Facing Reality],纽约:Springer Verlag,1970年版,第21页。]",
"大白蚁中的工蚁在建一座拱。每颗杆子都是通过添加小团的泥土和排泄物来建立的。在左边柱子的顶端可以看到一个工蚁在安放一个圆粪团。其它工蚁正叨着小团往柱子上爬,并把它们放置在增长的一端。当柱子达到一定的高度时,白蚁们开始以一定的倾角使它向相邻的柱子的方向伸展,这显然是在气味的引导下进行的。在背景上可以看到一个完成了的拱。[图立德·何达布勒画,摘自成尔逊的《昆虫杜会》,第230页。]",
"作者的“语义网络”的一个片段。 ",
"秩序与紊乱,艾舍尔作(蚀版画,1950)。",
"一个待调用的BlooP程序的结构。由于这个程序是自足的,每个过程定义中只能调用定义在它前面的过程。",
"盖奥尔格·康托尔",
"上和下,艾舍尔作(石版画,1947)。",
"TNT的“多重分叉现象”。TNT的每个扩充都有它自己的哥德尔句子,可以把这个句子或其否定添加进去,使得从TNT的每个扩充都生出一对进一步的扩充。这是一个直到无穷的过程。",
"龙,艾舍尔作(木刻,1952)",
"影子,马格里特作(1966)。",
"优雅状态,马格里特作(1959)。",
"烟草花叶病毒。[选自莱宁格[A.Leninger].《生物化学》[Biochemistry],纽约:Worth Publishers,1976年版。]",
"漂亮的俘虏,马格里特作(1947)。",
"\label {fig:81}十二个自噬的电视屏幕。如果13不是个素数的话,我会再加上二个的。",
"\label {fig:82}咏叹调和歌曲,马格里特作(1964)。",
"刘皓明绘。",
"",
"",
"一首自制的歌。",
"印符遗传密码。串中各个二元组按这种对应关系分别给十五种氨基酸(和一个标点符号)编码。",
"印符酶的三级结构",
"印符酶所拴的位置表。",
"“印符遗传学的中心法则”:缠结的层决结构”的一个例子。",
"DNA的四种基:腺嘌呤,鸟嘌呤,胞嘧啶,胸腺嘧啶。[摘自哈瓦那尔特和海因斯[Hanawalt; Haynes],《生命的化学基础》[The Chemical Basis of Life],旧金山:W. H. Freeman,1973年版,第142页。]",
"DNA结构象一架梯子。边上两条由脱氧核糖和磷酸盐的单元交替组成,横档以特殊方式配对——A配T,G配C——的基形成,基之间分别由两个或三个氧键相互连接。[摘自哈那瓦尔特和海因斯,《生命的化学基础》,第142页。]",
"DNA的双螺旋分子模型。[摘自维农·英格莱姆[Vernon M. Ingraml,《生物合成》[Biosynthesis],加利福尼亚州,Menlo Park: W. A. Benjamin,1972年版,第13页。]",
"遗传密码。信使RNA申中的每个三元组就这样为二十种氮基酸(及一个标点符号)编码。",
"从一些高分辨率的X光资料推断出的肌红蛋白的结构。大尺寸的“扭曲管道”形状是三级结构,内部的细螺旋——“阿尔法螺旋”——是二级结构。[选自莱宁格尔,《生物化学》。]",
"一段mRNA通过核糖体。漂浮在附近的是些tRNA分子,它们带有能被核糖体抓来并加进正在生成的蛋白质中的氨基酸。遗传密码作为一个整体包含在tRNA分子内。要注意,基对(A-U,C-G)在图中由卡在一起的字型表示。[斯科特·凯姆绘]",
"多核糖体。同一个mRNA串通过一个接一个的核糖体,就好象磁带依次通过若干台录音机。其结果是一批处于不同程度的正在生成的蛋白质:类似于相互错开的若干录音机所生成的音乐卡农。[选自莱宁格尔,《生物化学》。]",
"这里是一个更为复杂的图示,不只一个而是有若干个mRNA串,它们全都转录自同一个DNA串,并受到多核糖体作用,其结果是一个二排的分子卡农。[选自哈那瓦尔特和海因斯,《生命的化学基础》,第271页。]",
"中心法则映射。在两个基本的缠结构的层次系统——分子生物学层次系统和数理逻辑层次系统——之间的一个类比。",
"哥德尔编码。按照这个哥德尔配数方案,每个TNT符号有一个或几个密码子。小椭圆圈(括号)表明这个表怎样包含着第九章中的哥德尔配数表。",
"T4细菌病毒是一个蛋白质聚合体(a)。它的“头”是形如一种具有三十面的扁长的正十二面体的蛋白质膜,里面充满了DNA,头通过一个颈与尾部相连。尾部是包在一个可收缩的鞘内的空心芯子,安在四周有六条须根、下面有若干钉子的底盘上。钉子和须根使病毒牢牢地粘在细菌的细胞壁上(b)。鞘的收缩使芯子穿透细胞壁,病毒的DNA进入细胞。[选自哈那瓦尔特和海因斯,《生命的化学基础》,第230页。]",
"当病毒的DNA进入细菌之后,病毒感染就开始了。细菌的DNA被瓦解,而病毒的DNA得到复制。病毒结构蛋白的合成以及它们装配成病毒的过程一直在继续,直到撑破细胞,释放出微粒为止。[选自哈那瓦尔特和海因斯,《生命的化学基础》,第230页。]",
"T4病毒的形态发生通道有三个主分支,分别形成头、尾和尾须,然后结合起来形成完整的病毒微粒。[选自哈那瓦尔特和海凶斯,《生命的化学基础》,第237页。]",
"卡斯特罗瓦尔瓦,艾舍尔作(石板画,1930)。",
"湿利尼吠·拉玛奴衍及其神奇的印度旋律之一。",
"自然数的性质能在人类大脑和计算机程序中得到反映。这两个不同的描述因而能在一个适当的抽象层次上彼此对应。",
"大脑的符号层漂浮于神经原活动之上,从而反映了世界。不过能在计算机上模拟的那种神经原活动本身并不能产生思维,那得靠组织中的一些更高的层次。",
"使人热衷于从事人工智能研究的关键,在于这样一种观念:心智的符号层次能从它们的神经原基质上被“撇出'',并用诸如计算机的电子基质之类的方法实现。至于对大脑的复制需要做到什么深度,现在还完全不清楚。",
"大脑是理性的,而心智则可能不是。[作者绘]",
"“拣起一个大红方块。”[选自特里·维诺格拉德,《理解自然语言》,第8页。]",
"“找一块比你手里的那块大点的积木,把它搁到盒子里。”[选自维诺格拉德,《理解自然语言》,第9页。]",
"“你把两个红方块跟一个绿色的方块或是方锥摞成一摞,好吗?”[选自维诺格拉德,《理解自然语言》,第12页。]",
"在赛跑中获胜后的阿兰·图灵(1950年5月)。[摘自萨拉·图灵,《阿兰·图灵》]",
"“驴桥”的证明(由帕普斯[约公元300年]和吉伦特的程序[约公元1960年]所发现)。问题:证明等腰三角形二底角相等。解:由于该三角形是等腰的,故AP和AP'等长,因此三角形PAP'和三角形P'AP是全等的(边、边、边)。这就说是对应角是相等的。具体来说,两个底角是相等的。",
"芝诺为了从A到B所建立的无穷目标树。",
"一个用阿拉伯语写成的有意义的故事。[摘自埃·卡蔷比与莫·西哲尔梅西[A. Khatibi; M. Sijelmassi],《伊斯兰书法大观》[The Splen-dour of Islamic Calligraphy],纽约:Rizzoll,1976年版。]",
"心算,马格里特\pnum{1931}。",
"对“一个支撑着方锥的红立方体”的过程化表示。[引自罗杰·尚克和肯尼恩·科尔比,《思维的计算机模型和语言》,172页。]",
"邦加德问题51号。[摘自莫·邦加德,《模式识别》。]",
"邦加德问题47号。[摘自莫·邦加德,《模式识别》。]",
"邦加德问题91号。[摘自莫·邦加德,《模式识别》。]",
"邦加德问题49号。[摘自莫·邦加德,《模式识别》。]",
"一个求解邦加德问题的程序的概念网络片段。“结点”通过“连线”彼此相联,而连线又可能被联接。通过把连线看成动词,把它所联接的结点看成主语和宾语,你可以从图中抽取出一些句子来。",
"邦加德问题33号。[摘自莫·邦加德,《模式识别》。]",
"邦加德问题85~87号。[摘自莫·邦加德,《模式识别》]。",
"邦加德问题55号。[摘自莫·邦加德,《模式识别》。]",
"邦加德问题22号。[摘自莫·邦加德,《模式识别》。]",
"邦加德问题58号。[摘自莫·邦加德,《模式识别》。]",
"邦加德问题61号。[摘自莫·邦加德,《模式识别》。]",
"邦加德问题70~71号。[摘自莫·邦加德,《模式识别》。]",
"对话《螃蟹卡农》模式图。",
"",
"“树懒卡农”,选自巴赫的《音乐的奉献》。[乐谱是用唐纳德·伯德的程序“斯马特”印制的,美术字由陈春光设计。]",
"一个“作者三角形”。",
"\label {fig:135}画手,艾舍尔作(石版画,1948)",
"艾舍尔的《画手》的抽象示意图,上面部分似乎是个悖论,下面部分是它的解。",
"常识,马格里特作(1945--46)。",
"两个谜,马格里特作\pnum{1966}。",
"\label {fig:139}以烟为号。[作者绘]",
"如烟似梦。[作者绘]",
"人类的处境I,马格里特作\pnum{1933}。",
"\label {fig:142}画廊,艾舍尔作(石版画,1956)。",
"艾舍尔的《画廊》的抽象图示。",
"上图的一种压缩形式。",
"\fig{143}的进一步压缩形式。",
"\fig{143}进行压缩的另一种方式。",
"巴赫的“无穷升高的卡农”的六角形示意图。在使用谢泼德音调的条件下,它形成了一个不折不扣的闭合循环。",
"谢泼德音阶的两个完整周期的钢琴谱。每个音符的音量与其大小成正比,因此,当最上面的声部消失时,下面一个新的声部出现了[该图为唐纳德・伯德的程序“斯马特”所印出]",
"\label {fig:149}辞,艾舍尔作(蚀版画,1942)。",
"螃蟹的客人:查尔斯·巴比奇",
"螃蟹主题:C-Eb-G-Ab-B-B-A-B。把查·巴比奇的英文名字“Babbagc, C”倒过来念就是C-E-G-A-B-B-A-B]",
"《六部无插入赋格》的最后一页,选自巴赫《音乐的奉献》的初版。",
